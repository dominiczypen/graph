\documentclass[12pt, a4paper]{amsart}
\usepackage{amssymb}
\usepackage{amsthm}
\usepackage{amsfonts}
\usepackage{url}
%\usepackage{tikz}
\newtheorem{lemma}{\bf Lemma}[section]
\newtheorem{definition}[lemma]{\bf Definition}
\newtheorem{proposition}[lemma]{\bf Proposition}
\parindent = 0mm
\parskip = 2mm
\begin{document}
\title{Hadwiger's conjecture for hypergraphs}
\author{D.Z., \today}
%\address{Swiss Armed Forces, CH-3003 Bern, Switzerland}
%\email{dominic.zypen@gmail.com}
%----------------------
\maketitle
%----------------------
\section{Basic notions}
\subsection{Hypergraphs} A {\em hypergraph} $H=(V,E)$ consists of a
set $V$ and $E\subseteq {\mathcal P}(V)$, that is, $E$ 
consists of subsets of $V$ of arbitrary size. 

If $H = (V,E)$ is a hypergraph and $S\subseteq V$, we define
$$E|_S = \{e\cap S: (e\in E) \land (e\cap S \neq \emptyset)\}$$
and call $(S, E|_S)$ the {\em induced sub-hypergraph} of $H$.


\subsection{Connectedness} A hypergraph $H=(V,E)$ is {\em connected}
if for all $X \subseteq V$ with $\emptyset \neq X \neq V$ there
is $e\in E$ such that $$e\cap X \neq \emptyset \neq e
\cap (V\setminus X).$$

\subsection{Colouring} Let $H=(V,E)$ be a hypergraph and 
$\kappa\neq \emptyset$ be a cardinal.
Then a map $c:V\to \kappa$ is said to be a {\em colouring} if for
every $e\in E$ with $|e|\geq 2$ we have that
the restriction $c\restriction_e$ is non-constant. The {\em chromatic
number} $\chi(H)$ of $H$ is the smallest cardinal $\kappa$ such there
is a colouring $c:V\to \kappa$.

\subsection{Connected to each other}. If $H=(V,E)$ is a hypergraph
and $S_1, S_2\subseteq V$ are disjoint, we say they are {\em connected
to each other} if there is $e\in E$ such that $$e\cap S_1 \neq \emptyset
\neq e\cap S_2.$$

%----------------------------
\section{A form of Hadwiger's conjecture for hypergraphs}

Assume that $H=(V,E)$ is a hypergraph and let's assume 
$V\neq \emptyset \neq E$ to avoid pathologies. Let $\kappa$ be a
cardinal such that there is {\em no} colouring $c:V\to \kappa$. 
Then there is a collection ${\mathcal S}$ of  mutually
disjoint subsets with $|{\mathcal S}| = \kappa$ such that
\begin{enumerate}
	\item $(S, E|_S)$ is a connected hypergraph for each 
		$S\in {\mathcal S}$, and 
	\item whenever $S\neq T\in {\mathcal S}$ then $S, T$ are connected
	to each other.
\end{enumerate}
(In the graph context, this amounts to saying that there
is a complete minor of cardinality $\kappa$.)
\end{document}

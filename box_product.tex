% Generated using vi
% Author: Dominic van der Zypen
% Last modified 2021-10-20

\documentclass[12pt, a4paper]{amsart}
\usepackage{a4wide}
\usepackage{amssymb}
\usepackage{amsthm}
\usepackage{amsfonts}
\newtheorem{lemma}{\bf Lemma}[section]
\newtheorem{theorem}[lemma]{\bf Theorem}
\newtheorem{corollary}[lemma]{\bf Corollary}
\newtheorem{proposition}[lemma]{\bf Proposition}
\newtheorem{fact}[lemma]{\bf Fact}
\newtheorem{definition}[lemma]{\bf Definition}
\newtheorem{remark}[lemma]{\bf Remark}
\newcommand{\restrict}{\mbox{$\mid$}}
\newcommand{\Pow}{{\mathcal P}}
\newcommand{\Pz}{{\mathcal P}_2}
\newcommand{\card}{{\mathrm{card}}}

\begin{document}

%...... title
\title{Some notes on $\Box(\omega+1)^\omega$}

%...... authors
\author{Dominic van der Zypen, 2021-10-20}
%\address{M\&S Software Engineering, Morgenstrasse 129, CH-3018 Bern,
%Switzerland}
%\email{dominic.zypen@gmail.com}

%......MSC subject class
%\subjclass[2010]{05C15, 05C83}

%...... Abstract...
%\begin{abstract} We provide a new and very general formulation 
%of the Hadwiger conjecture in terms of hypergraphs (not necessarily 
%finite) and show that it implies the Hadwiger conjecture.
%\end{abstract}

%....... main()
\maketitle
\parindent = 0mm
\parskip = 2 mm
% . . . . . . . . . . . . . . . . .
\section{Topology on $\omega+1$}
We define the topology on $\omega+1$ to be $$\tau = {\mathcal P}(\omega)\cup 
\{A\subseteq (\omega+1): (\omega+1) \setminus A \text{ is finite}\}.$$
It is not hard to verify that this is a (compact) topology on $\omega+1$.
The space
$(\omega+1,\tau)$ is (homeomorphic to) the 1-point compactification of  the 
discrete topology on $\omega$.

% . . . . . . . . . . . . . . . . .
\section{The box product on a family of spaces $\{X_\lambda:\lambda \in \Lambda\}$}
Let $\{(X_\lambda,\tau_\lambda:\lambda \in \Lambda\}$ be a family of topological spaces.
Then the {\em box product topology} is the topological space on the 
base set $\prod_{\lambda \in \Lambda}X_\lambda$ generated by the basis
$$\{\prod_{\lambda \in \Lambda}U_\lambda: U_\lambda \in \tau_\lambda \text{ 
for all } \lambda \in \Lambda\}.$$
(It is a standard exercise to prove that the above set is indeed a basis.)

If all the spaces $X_\lambda$ are equal to space $X$, we denote the 
box product by $$\Box X^\Lambda.$$

% . . . . . . . . . . . . . . . . .
\section{Para- and metacompactness, and normalcy}
Let $(X,\tau)$ be a topological space. We call a collection ${\mathcal U}\subseteq \tau$ 
a {\em cover} if $\bigcup{\mathcal U} = X$. We call ${\mathcal U}$ {\em locally finite} 
if for every $x\in X$ there is an open neighborhood $U_0$ of $x$ such that 
$U_0$ intersects only finitely many members of $\mathcal U$. This is a way of 
saying that ${\mathcal U}$ is ``thinly spread'' over $X$, speaking with a lot 
of hand-waving. 

Moreover, $\mathcal U$ is {\em point-finite} if every $x\in X$ 
is only contained in finitely many members of $\mathcal U$. This is another, 
weaker notion of ``thinness'' than local finiteness, that is local finiteness 
implies point-finiteness.

A cover ${\mathcal V}$ of $X$ is
said to be a {\em refinement} of ${\mathcal U}$ if for every $V\in {\mathcal V}$
there is $U\in {\mathcal U}$ such that $V\subseteq U$.

Finally, $(X,\tau)$ is said to be {\bf paracompact} if every open cover 
has a locally finite refinement. For the weaker notion of {\bf metacompactness}, 
replace ``locally finite'' by ``point-finite''. 

A space is said to be {\bf normal} if disjoint closed sets can be separated 
by disjoint open sets. In $T_2$-spaces, normalcy and paracompactness is equivalent.

% . . . . . . . . . . . . . . . . .
\section{Is $\Box(\omega+1)^\omega$ paracompact?}
This is the big open question we are trying to think about, and I find it
more accessible in some way than the equivalent question about normalcy, despite 
remembering that we treated some cases of disjoint closed sets being able 
to be separated by disjoint open sets.

(Is it known whether $\Box(\omega+1)^\omega$ is metacompact?)

% . . . . . . . . . . . . . . . . .
\section{Introductory thoughts}
As an entry point, let's put an observation without proof (dangerous!! Will 
write one later on, because ``it's obvious'' is always dangerous).

Let $X = \Box(\omega+1)^\omega$, and let $\bar{\omega}$ be the constant 
$\omega$-sequence in $X$. For $n\in\omega$, let $\uparrow n = \{(\omega+1)\setminus n\}$. 

\begin{lemma} If $U\subseteq X$ is an open neighborhood of $\bar{\omega}$, 
then there is a {\em monotone map} $f:\omega\to\omega$ such that 
$$\bar{\omega}\in \prod_{n\in\omega}(\uparrow f(n))\subseteq U.$$
\end{lemma}

{\bf Exercise I would like to solve as a first step.} 
Given monotonically increasing $f:\omega\to\omega$ with $f(0)>0$, 
consider the following cover of $X$. Let ${\mathcal U}$ consist of
\begin{itemize}
    \item $\prod_{n\in\omega}(\uparrow f(n))$, covering $\bar{\omega}$, 
    \item the singletons $\{a\}$ for $a\in \omega\to\omega$, and finally 
    \item $\prod U_x$ covering $x\in X\setminus \big(\{\bar{\omega}\} 
    \cup \omega^\omega\big)$
    where $U_x$ = $\{x(n)\}$ if $x(n)\in\omega$, and $U_x = \uparrow f(n)$ 
    if $x(n) = \omega$. 
\end{itemize}
Questions: Is ${\mathcal U}$ locally finite? Point-finite? If no, does it have 
such a refinement?

(Maybe this is a very easy exercise, but variations on the cover sets in 
the third bullet point could make it interesting.)
\end{document}

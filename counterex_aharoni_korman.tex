% Generated using vi
% Author: Dominic van der Zypen
% Last modified 2022-04-28

\documentclass[12pt]{amsart}
\usepackage{amssymb}
\usepackage{amsthm}
\usepackage{amsfonts}
\newtheorem{lemma}{\bf Lemma}[section]
\newtheorem{observation}[lemma]{\bf Observation}
\newcommand{\Pow}{{\mathcal P}}
\newcommand{\beweisende}{{\hfill $\Box$}}
\newcommand{\card}{{\mathrm{card}}}

\begin{document}

%...... title
\title{Counterexample to a conjecture of Aharoni and Korman}

%...... authors
\author{Dominic van der Zypen}
\address{Swiss Armed Forces, CH-3003 Bern,
Switzerland}
\email{dominic.zypen@gmail.com}

%......MSC subject class
\subjclass[2010]{05C15, 05C83}

%...... Abstract...
\begin{abstract}
	In \cite{Ah}, Aharoni and Kleitman conjectured that
	any hypergraph with finite edges only has a strongly
	minimal cover. We present a counterexample.
\end{abstract}
\parindent = 0mm
\parskip = 2 mm
% . . . . . . . . . . . . . . . . .
%--------------------------
\maketitle

\section{Definitions}

A {\em hypergraph} is a pair $H=(V,E)$
where $V$ is a set and $E\subseteq \Pow(V)$. The elements of
$E$ are called {\em edges}.

An {\em (edge) cover} of a hypergraph is a subset of 
$K\subseteq E$ such that $\bigcup K = V$. 

We say that a cover $M \subseteq E$ is {\em strongly minimal} if for every
cover $K\subseteq E$ we have $\card(M\setminus K) \leq \card(K\setminus M)$.

\section{The counterexample}
Conjecture 5.4 of \cite{Ah} says that 
any hypergraph with finite edges only has a strongly
minimal cover. 
\begin{lemma} Let $H= (\omega, E)$ where $E$ is the collection of 
	finite subsets of $\omega$. Then $H$ has no strongly minimal
	cover.
\end{lemma}
{\em Proof.} Let $K\subseteq E$ be any cover of $\omega$. We want to 
show that $K$ is not strongly minimal. As $\bigcup K = \omega$, and $K$
consists of finite subsets of $\omega$, we
can pick two
different non-empty elements $a, b\in K$. Let $$K^* = 
\big(K \setminus \{a, b\}\big) \cup \{a \cup b\}.$$
Clearly $K^*$ is a cover of $\omega$ and $K \setminus K^* = \{a, b\}$,
but $K^*\setminus K$ is either empty (if $a\cup b$ already was an element 
of $K$), or consists of $1$ element only, namely $a\cup b$. 

So we have $$\card(K\setminus K^*) = 2 > 1 \geq \card(K^* \setminus K).$$
Therefore, $K$ is not strongly minimal. \beweisende
%--------------------------
\section{Acknowledgements}
I want to thank Jonathan David Farley for fruitful discussions
on this topic.
%................... bibliography .................
{\footnotesize
\begin{thebibliography}{99}
%\bibitem{Er} \Erd, Paul, {\it On the combinatorial problems which I would
%most like to see solved}, Combinatorica {\bf 1} (1981), 25--42.
%\bibitem{Ha} Hugo Hadwiger, {\it \"Uber eine Klassifikation der 
%Streckenkomplexe}, Vierteljschr.~Naturforsch.~Ges.~Z\"urich, 
%{\bf 88} (1943), 133--143.
\bibitem {Ah} Ron Aharoni, Vladimir Korman, {\it
Greene-Kleitman's Theorem for Infinite Posets}, Order
({\bf 9}) (1992),  245--253.
%\bibitem{me} Dominic van der Zypen, {\it Hadwiger's conjecture for graphs with 
%infinite chromatic number}, Advancement and Development in Mathematical
%Sciences {\bf 4}, 2013, issue 1\&2, 1--4.
%--------------------------
\end{thebibliography}
}
\end{document}

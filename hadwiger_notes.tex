\documentclass[12pt, a4paper]{amsart}
\usepackage{amssymb}
\usepackage{amsthm}
\usepackage{amsfonts}
\usepackage{url}
%\usepackage{tikz}
\newtheorem{lemma}{\bf Lemma}[section]
\newtheorem{definition}[lemma]{\bf Definition}
\newtheorem{proposition}[lemma]{\bf Proposition}
\parindent = 0mm
\parskip = 2mm
\begin{document}
\title{Some notes on the Hadwiger conjecture}
\author{D.Z., \today}
%\address{Swiss Armed Forces, CH-3003 Bern, Switzerland}
%\email{dominic.zypen@gmail.com}
%----------------------
\maketitle
%----------------------
\section{Introductory remarks}
Hugo Hadwiger formulated his conjecture in 1943. 
Some historical background on this conjecture can be
found on Wikipedia using the search terms 
\begin{quote}{\tt wiki hadwiger conjecture graph theory}
\end{quote}
Erd\H{o}s considered this conjecture to be one of the 
top three problems he would like to see solved in his 
lifetime.
%----------------------
\section{Basic notions}
\subsection{Graphs} For any set $X$ we let $$[X]^2 = \big\{\{x,y\}: (x, y\in X)
\land (x\neq y)\big\}.$$ A {\em graph} is a tuple $G =(V,E)$
where $E \subseteq [V]^2$. 

If $G=(V,E)$ is a graph and $v\in V$, then the {\em neighborhood}
of $v$ is defined by $$N(v) = \{w\in V:\{v,w\}\in E\}.$$ Note
that we always have $v\notin N(v)$. The cardinal 
$\text{card}(N(v)) = |N(v)|$ is said to be the {\em degree} of $v$,
denoted by $\text{deg}(v)$.

\subsection{(Induced) subgraphs} For $G=(V,E)$ and $S\subseteq V$
we define the induced subgraph of $S$ by $$(S, E\cap [S]^2).$$
For $v\in V$ we denote the induced subgraph on $V\setminus\{v\}$ by
$G\setminus\{v\}$.


\subsection{Hypergraphs} A {\em hypergraph} $H=(V,E)$ consists of a
set $V$ and $E\subseteq {\mathcal P}(V)$, that is, $E$ 
consists of subsets of $V$ of arbitrary size. Obviously,
a graph is a special kind of hypergraph.

\subsection{Colouring} We define a notion of colouring for {\em 
hypergraphs} that agrees with the notion of graph coloring whenever
the hypergraph at hand is a graph. Let $H=(V,E)$ be a hypergraph and 
$\kappa\neq \emptyset$ 
be a cardinal.
Then a map $c:V\to \kappa$ is said to be a {\em colouring} if for
every $e\in E$ with $|e|\geq 2$ we have that
the restriction $c\restriction_e$ is non-constant. The {\em chromatic
number} $\chi(H)$ of $H$ is the smallest cardinal $\kappa$ such there
is a colouring $c:V\to \kappa$.

A few (non-trivial\footnote{Disclosure: I find them hard and couldn't
do all, but some or all of these points could be trivial}) 
things on hypergraph colouring:
\begin{enumerate}
\item If $\kappa$ is an infinite cardinal and $E$ is a collection 
of subsets of cardinality $\kappa$, and also we have $|E| = \kappa$,
then $\chi(\kappa, E) = 2$. 
\item Let $\omega$ be the first infinite cardinal. Given $n\in 
\omega\cup\{\omega\}$, is there $E\subseteq {\mathcal P}(\omega)$
such that $\chi(\omega, E) = n$?
\item Determine $\chi(\mathbb{N}, E)$ where 
\begin{enumerate}
\item $E = \big\{\{a,b,a+b\}: a, b\in \mathbb{N}\big\}$,
\item $E = $ collection of subsets of $\mathbb{N}$ such 
that its members are pairwise relatively prime.
\end{enumerate}
\item What is $\chi(\mathbb{R}, E)$ where $E$ is the collection
	of infinite Lebesgue-measurable sets? (We have $|E| >
		2^{\aleph_0} = |\mathbb{R}|$ if I am not mistaken.)
\end{enumerate}
\subsection{Point contraction} If $v\neq w\in V$ we can ``contract'' 
$v$ and $w$ in the following way. Consider $V' = V\setminus \{w\}$
and set $$E' = (E \cap [V']^2) \cup \big\{\{v,x\}: x\in N(w)
\setminus\{v\}\big\}.$$
Then we denote the graph $(V', E')$ by $G/\{v,w\}$.
%----------------------
\section{Connectedness}
Usually, a graph $G=(V,E)$ is said to be {\em connected} if there is
a path between any two vertices (points) $v, w\in V$. The following
notion of connectedness seems more general, and it can be
used in the context of hypergraphs too:
\begin{definition}
$G = (V,E)$ is {\em connected} if for all $X\subseteq V$ with 
$\emptyset \neq X \neq V$ there is $e\in E$ with 
	$$X \cap e \neq \emptyset \neq (V\setminus X)\cap e.$$
\end{definition}
I was convinced that for infinite graphs the above definition
was {\em weaker}, but it turns out there is a short inductive
argument showing that this is equivalent to the path definition
we are used to.
%----------------------
\section{Complete minors (no age requirement), and the Hadwiger number}
\begin{definition}
If $S, T$ are disjoint subsets of $V$ where $G=(V,E)$ is a graph,
	we say $S,T$ are {\em connected to each other} if there
	is $s\in S, t\in T$ such that $\{s,t\} \in E$.
\end{definition}
A {\em complete minor} of $G=(V,E)$ is a collection ${\mathcal S}$ 
of {\em connected}, non-empty, subsets of $V$ such that
\begin{enumerate}
	\item whenever $S\neq T\in {\mathcal S}$ then $S\cap T = \emptyset$, and
	\item whenever $S\neq T\in {\mathcal S}$ then $S, T$ are connected
		to each other.
\end{enumerate}
The {\em Hadwiger number} $\eta(G)$ of a graph $G=(V,E)$ is the supremum
of the cardinalities that a complete minor can have, i.e.
$$\eta(G) = \sup\{|{\mathcal S}|: {\mathcal S}\subseteq {\mathcal P}(V) \text{ is
a complete minor}\}.$$

%----------------------
\section{Hadwiger's conjecture}
Hadiwger proposes the following easy connection between the
chromatic number $\chi(G)$ of a graph, and its Hadwiger number:

\vspace*{2mm}
\fbox{{\bf Hadwiger's conjecture.} $\chi(G) \leq \eta(G)$ for all graphs $G$.}

Some remarks:

\begin{enumerate}
\item As soon as $E \neq \emptyset$ for a graph $G=(V,E)$,
	we have $\eta(G) \geq 2.$
\item Note that even for $\chi(G)=2$, the value of $\eta(G)$ can 
	be arbitrarily large: Let $\kappa\geq 2$ be a cardinal (finite
		or infinite). Then $\eta(K_{\kappa,\kappa}) = \kappa$
		where $K_{\kappa, \kappa}$ is the complete bipartite
		graph on $2\cdot \kappa$ points. Obviously
		$\chi(K_{\kappa,\kappa}) = 2$.
\end{enumerate}

%----------------------
\section{Hadwiger's conjecture in the infinite}
The situation for graphs with infinite chromatic number is much
clearer than for finite graphs\footnote{this also covers infinite graphs 
with {\em finite chromatic number} because of the De Bruijn - Erd\H{o}s
theorem stating every such graph has a {\em finite} subgraph
with the same chromatic number}.
\begin{proposition} If $G=(V,E)$ is such that $\chi(G)\geq \aleph_0$, 
	then $\chi(G)\leq \eta(G)$.
\end{proposition}
This is a corollary of the main result of 
\url{https://arxiv.org/pdf/1312.2829.pdf}.

However, we can reformulate Hadwiger's conjecture as follows:

{\bf Hadwiger Version 2.} Any graph $G$ has a complete minor of 
cardinality $\chi(G)$.

Version 2 is equivalent to the original statement for FINITE graphs,
but is is {\em false} for graphs with infinite chromatic number:

Let $G = \bigcup_{n\in \mathbb{N}}K_n$ be the disjoint union of copies of
the complete graph $K_n$. Then $\chi(G)$ is infinite, but $G$ has
only finite complete minors (but of arbitrarily large finite size).

%----------------------
\section{Properties of a minimal hypothetical counterexample}
\subsection{Criticality} A graph $G=(V,E)$ is said to be {\em critical}
if $\chi(G\setminus \{v\}) < \chi(G)$ for all $v\in V$.
For the remainder of this section let $n_0$ be the {\em minimum
cardinality} $|V|$ of any graph $G_0=(V,E)$ that is a counterexample
to Hadwiger, i.e. has $$\eta(G_0) < \chi(G_0).$$
\begin{proposition} 
$G_0$ is critical.
\end{proposition}
{\em Proof.} Otherwise, take $v\in V$ with $\chi(G_0\setminus 
\{v\} = \chi(G_0)$.
Then $\eta(G_0\setminus\{v\}) \leq \eta(G_0) < \chi(G_0) = \chi(G_0
	\setminus\{v\}$ so that $G_0\setminus \{v\}$ is
	another counterexample to Hadwiger with $|V\setminus\{v\}| =
	n_0-1 < n_0$, contradicting minimality of $n_0$. 
	\hfill{$\Box$}

Note that every critical graph is connected.

Now we revisit contractions (see ``Basic notions''). 

Let $v\neq w\in V$ such that $\{v,w\}\notin E$.
\begin{proposition}
If $G$ is any graph and $v\neq w\in V$ with $\{v,w\}\notin E$, then
	$$\chi(G/\{v,w\}) \geq \chi(G).$$
\end{proposition}

Let's go back to our minimal Hadwiger counterexample $G_0$.

\begin{proposition} \label{contractprop}
Whenever $v\neq w \in V(G_0)$ and $\{v,w\}\notin E$, then
$\eta(G_0/\{v,w\}) > \eta(G_0)$.
\end{proposition}
{\em Proof.} Suppose $\eta(G_0/\{v,w\}) \leq \eta(G_0)$. Then since
$\chi(G_0/\{v,w\} \geq \chi(G_0)$ we get $\eta(G_0/\{v,w\}) < 
\chi(G_0/\{v,w\})$, so $G_0/\{v,w\}$ is a Hadwiger counterexample
with a smaller number of vertices than $G_0$, contradicting minimality.
\hfill{$\Box$}.

\begin{proposition} \label{edgeprop}
If $G=(V,E)$ is any graph and $v\neq w\in V$ with $\{v,w\} \notin E$,
then $$\eta(G/\{v,w\}) \leq \eta(G \cup \{v,w\}),$$
where $G\cup \{v,w\}$ denotes the graph $(V, E\cup\{v,w\})$.
\end{proposition}

Together, propositions \ref{contractprop} and \ref{edgeprop}
imply that $G_0$ must have the following property:
\begin{quote}
	(E): Whenever a new edge is added in $G_0$, then the
	Hadwiger number increases.
\end{quote}

So $G_0$ has property (E), and it is critical. I asked
Paul Seymour\footnote{Quite a famous combinatorialist, we exchange
e-mail like once or twice a year} to come up with {\em any} finite 
critical graph $G_1$ having property (E), and he wrote after
a while that he wasn't able to.

If some such $G_1$ could be found, this wouldn't refute Hadwiger
but could give insight as to what kind of examples could be interesting
regarding Hadwiger.

Any proof that shows no critical graph with property (E) can exist
would prove the Hadwiger conjecture.

%---------------------------
\section{Extending to hypergraphs?}
We can define connected hypergraphs, see ``Basic notions'',
and we can extend the concept of colouring to hypergraphs.
Maybe we can define a complete minor in a hypergraph.

{\bf Question.} Is there a cool way to formulate Hadwiger's conjecture
in terms of hypergraphs?
\end{document}
